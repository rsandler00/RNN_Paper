\documentclass[11pt,a4paper,final]{article}
\usepackage[utf8]{inputenc}
\usepackage[english]{babel}

\usepackage{bm}             %for bold math symbols
\usepackage{latexsym}
\usepackage{amssymb}
%\usepackage[]{natbib}

\usepackage{csquotes}
\usepackage[backend=bibtex8,natbib=true,style=authoryear,labelnumber]{biblatex}
\addbibresource{NET.bib}
\addbibresource{C:/Users/Roman/Documents/Thesis/THESIS_Latex/biblio.bib}
\DeclareFieldFormat{labelnumberwidth}{[#1]}
\defbibenvironment{bibliography}  % from numeric.bbx
  {\list
    {\printtext[labelnumberwidth]{%
      \printfield{prefixnumber}%
      \printfield{labelnumber}}}
    {\setlength{\labelwidth}{\labelnumberwidth}%
        \setlength{\leftmargin}{\labelwidth}%
        \setlength{\labelsep}{\biblabelsep}%
        \addtolength{\leftmargin}{\labelsep}%
        \setlength{\itemsep}{\bibitemsep}%
        \setlength{\parsep}{\bibparsep}}%
        \renewcommand*{\makelabel}[1]{\hss##1}}
  {\endlist}
  {\item}
\DefineBibliographyStrings{english}{%
  bibliography = {References},
}

\usepackage[font={small,it}]{caption}   %for image captions
\usepackage[margin=1in]{geometry}
\usepackage{hyperref}       %for linking in contents, refs, & images
\usepackage{authblk}

\usepackage{epsf}           %for .EPS graphics inclusion
\usepackage{graphicx}
\usepackage{epstopdf}

\usepackage[font={small,it}]{caption}   %for image captions
\graphicspath{{C:/Users/Roman/Documents/NET/figures/}}
\RequirePackage{amsopn}
%\RequirePackage{affronts}
\RequirePackage{amsmath}
\usepackage{lineno}

\usepackage{abstract} % Allows abstract customization
\renewcommand{\abstracttextfont}{\normalfont\small\itshape} %Set the abstract itself to small italic text

\title{\vspace{-30mm}\fontsize{14pt}{1pt}\textbf{
Designing Patient-Specific Optimal Neurostimulation Patterns for Seizure Suppression}} % Article title

\author[1,2]{Roman A. Sandler     \thanks{Corresponding Author: rsandler00@gmail.com}}
\author[3]{Kunling Geng         }   %   kgeng@usc.edu
\author[3]{Dong Song            }   %   dsong@usc.edu
\author[4]{Robert E. Hampson    }   %   rhampson@wakehealth.edu
\author[5]{Mark R. Witcher      }   %   mark.russell.witcher@emory.edu
\author[4]{Sam A. Deadwyler     }   %   sdeadwyl@wakehealth.edu
\author[3]{Theodore W. Berger   }   %   berger@usc.edu
\author[3]{Vasilis Z. Marmarelis}   %   vzm@usc.edu
\affil[1]{Department of Physics \& Astronomy, University of California, Los Angeles, Los Angeles, CA, USA}
\affil[2]{W. M. Keck Center for Neurophysics, University of California, Los Angeles, Los Angeles, CA, USA}
\affil[3]{Department of Biomedical Engineering, University of Southern California, Los Angeles, CA, USA}
\affil[4]{Department of Physiology \& Pharmacology, Wake Forest University, Winston-Salem, NC, USA} %27106
\affil[5]{Department of Neurosurgery, Wake Forest University, Winston-Salem, NC, USA}
\renewcommand\Authands{ \& }

%\linenumbers
\begin{document}

\newcommand{\nn}{24}    %total amount of neurons
\newcommand{\fit}{170}    %optimal FIT freq
\newcommand{\rit}{130}    %optimal RIT freq
\newcommand{\len}{250}   %length of applied DBS in ms
\newcommand{\success}{92} % % of seizures abated by optimal stim

\newcommand{\sig}{18}   % # of significnat models
\newcommand{\sparse}{22.83} % % of significnat connections

%\keywords{epilepsy, DBS, neurostimulation, seizure, graph theory, hippocampus, GLM}
%\reviewers{Uri T Eden, tzvi@bu.edu Theoden Netoff tnetoff@umn.edu  Wilson Truccolo Wilson_Truccolo@brown.edu Fabrice Wendling  fabrice.wendling@univ-rennes1.fr}


\maketitle % Insert title

\begin{abstract}
Neurostimulation is a promising therapy for abating epileptic seizures.
However, it is extremely difficult to identify optimal stimulation patterns experimentally.
%In this study we use nonlinear statistical modeling to reconstruct the unique connectivity and dynamics of \nn{} neurons recorded from human hippocampus.
In this study human recordings are used to develop a functional \nn{} neuron network statistical model of hippocampal connectivity and dynamics. 
Spontaneous seizure-like activity is induced \textit{in-silico} in this reconstructed neuronal network.
The network is then used as a testbed to design and validate a wide range of neurostimulation patterns.
Commonly used periodic trains were not able to permanently abate seizures at any frequency.
A simulated annealing global optimization algorithm was then used to identify an optimal stimulation pattern which successfully abated \success{}\% of seizures.
Finally, in a fully responsive, or "closed-loop" neurostimulation paradigm, the optimal stimulation successfully prevented the network from entering the seizure state.
We propose that the framework presented here for algorithmically identifying patient-specific neurostimulation patterns can greatly increase the efficacy of neurostimulation devices for seizures.
\end{abstract}

\section*{Publication Statement}
None of this material has been published elsewhere




\end{document} 